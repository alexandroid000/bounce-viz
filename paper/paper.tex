\documentclass[letterpaper, 10 pt, conference]{ieeeconf}  % Comment this line out

\IEEEoverridecommandlockouts                              % This command is only
                                                          % needed if you want to
                                                          % use the \thanks command
\overrideIEEEmargins

\usepackage{amsmath}    % need for subequations
\usepackage{graphicx}   % need for figures
\usepackage[font=footnotesize,labelfont=footnotesize]{caption}
\usepackage{verbatim}   % useful for program listings
\usepackage{color}      % use if color is used in text
\usepackage{subfigure}  % use for side-by-side figures
\usepackage{hyperref} % use for hypertext links, including those to external documents and URLs
\usepackage{multirow}
\usepackage{rotating}
\usepackage{array}
\usepackage{xspace}
\usepackage{indentfirst}
\usepackage{amssymb}
\usepackage{float}
\usepackage{algorithm}
\usepackage{algorithmic}
\usepackage{comment}
\usepackage{sidecap}
\usepackage{booktabs}

\title{Mobile Robots in Polygons: a Combinatorial Data Structure for
Navigation, Localization, and Coverage}

\author{Alexandra Q. Nilles, Samara Ren% <-this % stops a space
%\thanks{This work was partially supported by National Science Foundation (award numbers CMMI-1100579 and IIS-1302393).}
}

\begin{document}


\maketitle

%%%%%%%%%%%%%%%%%%%%%%%%%%%%%%%%%%%%%%%%%%%%%%%%%%%%%%%%%%%%%%%%%%%%%%%%%%%%%%%%
%\begin{abstract}
%
%\end{abstract}

{\small
\begin{center}
\begin{quotation}
``Geometry is not true, it is advantageous." \\
\hfill    --- Henri Poincar\'e
\end{quotation}
\end{center}
}

%%%%%%%%%%%%%%%%%%%%%%%%%%%%%%%%%%%%%%%%%
\section{Introduction} 


\subsection{Related Work}

\section{Combinatorial Visibility Data Structures}

\subsection{Background}

\begin{itemize}
\item Visibility diagrams and link diagrams
\item Partial local sequence and edge-edge visibility equivalence
\end{itemize}

\subsection{Bounce Equivalence Diagram?}

\begin{itemize}
\item Big O complexity of construction, and size of resulting graph
\item Examples
\end{itemize}

\section{Task Formulation on BED}

\subsection{Tasks}

\begin{itemize}
\item Navigation
\item Patrolling - show correspondence to previous results on limit cycles
\item Localization - incorporate "Localization with Limited Sensing"
\end{itemize}

\subsection{Incorporating Sensor Models}

\begin{itemize}
\item Laser Beams
\item Pebbles
\item Cameras
\end{itemize}


\section{Toward a Hierarchy of Bouncing Robots}




\section{Open Questions and Future Work}

randomized plans?

\section{Conclusion}



\bibliographystyle{IEEEtran}
\bibliography{refs}

\end{document}
