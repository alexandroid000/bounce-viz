\documentclass[letterpaper, 10 pt, conference]{ieeeconf}  % Comment this line out

\IEEEoverridecommandlockouts                              % This command is only
                                                          % needed if you want to
                                                          % use the \thanks command
\overrideIEEEmargins

\usepackage{amsmath}    % need for subequations
\usepackage{graphicx}   % need for figures
\usepackage[font=footnotesize,labelfont=footnotesize]{caption}
\usepackage{verbatim}   % useful for program listings
\usepackage{color}      % use if color is used in text
\usepackage{subfigure}  % use for side-by-side figures
\usepackage{hyperref} % use for hypertext links, including those to external documents and URLs
\usepackage{multirow}
\usepackage{rotating}
\usepackage{array}
\usepackage{xspace}
\usepackage{indentfirst}
\usepackage{amssymb}
\usepackage{float}
\usepackage{algorithm}
\usepackage{algorithmic}
\usepackage{comment}
\usepackage{sidecap}
\usepackage{booktabs}

\title{Mobile Robots in Polygons: a Combinatorial Data Structure for
Navigation, Localization, and Coverage}

\author{Alexandra Q. Nilles, Samara Ren% <-this % stops a space
%\thanks{This work was partially supported by National Science Foundation (award numbers CMMI-1100579 and IIS-1302393).}
}

\begin{document}


\maketitle

%%%%%%%%%%%%%%%%%%%%%%%%%%%%%%%%%%%%%%%%%%%%%%%%%%%%%%%%%%%%%%%%%%%%%%%%%%%%%%%%
%\begin{abstract}
%
%\end{abstract}

{\small
\begin{center}
\begin{quotation}
``Geometry is not true, it is advantageous." \\
\hfill    --- Henri Poincar\'e
\end{quotation}
\end{center}
}

%%%%%%%%%%%%%%%%%%%%%%%%%%%%%%%%%%%%%%%%%
\section{Introduction} 


\subsection{Related Work}

\section{Data Structure}

\subsection{Motivation}

\begin{itemize}
\item Visibility diagrams and link diagrams
\item Partitioning boundary into natural equivalence classes
\end{itemize}

\subsection{Construction}

\begin{itemize}
\item Big O complexity of construction, and size of resulting graph
\item Example with obstacles
\end{itemize}

\section{Navigation}

\section{Minimal Filters and Plans}

\begin{itemize}
\item Navigation
\item Patrolling
\item Localization
\end{itemize}

\section{Incorporating Sensor Models}

\subsection{Laser Beams}

\subsection{Pebbles}

\subsection{Cameras}

\section{Randomized Plans}

\section{Open Questions}

\section{Conclusion}

%\begin{figure}
%\centering
%\includegraphics[trim=15 780 300 0,clip,scale=0.7]{design.pdf}
%\includegraphics[trim=15 700 300 80,clip,scale=0.7]{design.pdf}
%\includegraphics[trim=15 620 300 160,clip,scale=0.7]{design.pdf}
%\includegraphics[trim=15 540 300 240,clip,scale=0.7]{design.pdf}
%\includegraphics[trim=15 460 300 320,clip,scale=0.7]{design.pdf}
%\caption{Interaction modalities of computer-aided design of robots are distinguished by where the
%representation of the robot design is stored (in a computer or in a human
%brain), and by the direction and effect of queries.
%[First]~A model of traditional synthesis.
%[Second]~Automated design, where the automated system queries a human to resolve
%incompatibilities or underspecified components of the design.
%[Third]~Interactive design tools to answer human designer queries, such as,
%``what's the lightest sensor that can detect the color blue?", or ``if I 3D print this
%wheel, will it be strong and light enough for my robot car?"
%[Fourth]~Interactive design tools, where human questions inform the formalized
%solution. For example, a human may query the system, ``what happens if I make the
%legs twice as long?" which will change the current design solution on the
%computer.
%[Fifth]~Socratic model: an automated system which asks questions and/or provides
%suggestions to inspire a human designer.
%%
%(This figure combines contributions from Andrea Censi, Ankur Mehta, and
%the present authors)\label{fig:design}.} \end{figure}



\bibliographystyle{IEEEtran}
\bibliography{refs}

\end{document}
