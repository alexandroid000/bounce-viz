\documentclass[11pt]{article}%

\usepackage{styles/style}

\begin{document}
\section{Contraction ratio with uncertainty}
\[\frac{f_{-}(x, \theta)}{x} = \frac{\sin(\theta + \delta)}{\sin(\theta + \delta - \phi)} \eqqcolon a\]
\[\frac{f_{+}(y, \theta)}{y} = \frac{\sin(\theta - \delta)}{\sin(\theta - \delta - \phi)} \eqqcolon b\]
\[UC(\theta, \phi) = \frac{f_{+}(y, \theta)- f_{-}(x, \theta)}{y-x} = \frac{ay-bx}{y-x} = a+(a-b)\frac{x}{y-x}\]

Since $a > b$, if $y-x$ remains the same, then as $x$ increases, the contraction ratio with uncertainty increases; if $x$ remains the same, then as $y$ increases, the contraction ratio with uncertainty decreases.

\begin{tikzpicture}
\coordinate [label = 150:$O$](o) at (-5.5, -2.4);
\coordinate [label = -90:$x$](x) at (0.8, -2.4);
\coordinate [label = -90:$y$](y) at (3.3, -2.4);
\coordinate (z) at (5, -2.4);

\coordinate [label = 150:{$f(x, \theta)$}](fx) at (-0.9, 0.6);
\coordinate [label = 90:{$f(y, \theta)$}](fy) at (0.9, 1.8);
\coordinate (fz) at (4.1, 4);

\coordinate [label = 150:{$f_{-}(x, \theta)$}](f-x) at (-1.8, 0);
\coordinate [label = 90:{$f_{+}(x, \theta)$}](f+x) at (0.1, 1.3);
\coordinate [label = 150:{$f_{-}(y, \theta)$}](f-y) at (-0.4, 1);
\coordinate [label = 150:{$f_{+}(y, \theta)$}](f+y) at (2.2, 2.7);
\draw [line width=0.5mm](o) -- (z);
\draw [line width=0.5mm](o) -- (fz);
\draw pic["$\phi$",draw=red,<->,angle eccentricity=1.2,angle radius=1cm] {angle=z--o--fz};
\draw (x) -- (fx);
\draw (y) -- (fy);
\draw (x) -- (f-x);
\draw (y) -- (f+y);
\draw[dotted] (x) -- (f+x);
\draw[dotted] (y) -- (f-y);
\draw pic["$\theta$",draw=red,<->,angle eccentricity=1.2,angle radius=1cm] {angle=z--x--fx};
\draw pic["$\theta$",draw=red,<->,angle eccentricity=1.2,angle radius=1cm] {angle=z--y--fy};
\draw pic["$\delta$",draw=red,<->,angle eccentricity=1.2,angle radius=1.3cm] {angle=fx--x--f-x};
\draw pic["$\delta$",draw=red,<->,angle eccentricity=1.2,angle radius=1.3cm] {angle=f+y--y--fy};


\end{tikzpicture}
\end{document}